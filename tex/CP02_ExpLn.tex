\documentclass[CJK,13pt]{beamer}
\input{macros.tex}
\input{titlepage.tex}
  \date{}
  \begin{document}
  \bch
  \tpage{2}{指数和自然对数}

  \secpage{指数函数}{$$ e^x $$}

\begin{frame}

  \addfig{1}{mengli.jpg}
  
  假设银行存款年利率为 $1\%$,你把一块钱存入银行,一万年后取出时是多少钱?
\end{frame}

  

\begin{frame}
  快速估算
  $$ 1.02^{10}, 1.02^{20}, 1.02^{30}, 1.02^{40}, 1.02^{50}$$
  并把结果和(计算器摁出来的)精确结果相比较。
\end{frame}


\begin{frame}
  \frametitle{知识点}
  \tbox{$$\lim_{n\rightarrow\infty}\left(1+\frac{1}{n}\right)^n = e = 2.718 \ldots $$}
  或者更一般地
  \tbox{$$\lim_{x\rightarrow 0^+}\left(1+x\right)^{\frac{1}{x}} = e = 2.718 \ldots $$}  
\end{frame}



\begin{frame}
  \frametitle{思考题}
  利用二项式展开定理,说明

  $$e = \sum_{n=0}^\infty \frac{1}{n!} $$
\end{frame}



\begin{frame}
  \frametitle{指数函数 $e^x$}
  以 $e$ 为底的幂次 $e^x$ 称为指数函数,满足:
  \tbox{$$e^x > 0, \forall x\in \mathcal{R} $$}        
  \tbox{$$e^{0} = 1 $$}      
  \tbox{$$e^{x+y} = e^x e^y$$}
  \tbox{$$e^{x-y} = \frac{e^x}{e^y}$$}
  \tbox{$$e^{-x} = \frac{1}{e^x}$$}  
  
\end{frame}



\begin{frame}
  \frametitle{知识点}
  \tbox{当 $|x|\ll 1$ 时,$$e^x \approx 1+x$$}
\end{frame}



\begin{frame}
 在函数曲线 $y=e^x$ 上过点 $(a, e^a)$ 作切线,切线的斜率是多少?
\end{frame}


\secpage{自然对数}{$$\ln x \equiv \log_e x$$}


\begin{frame}
  \frametitle{自然对数}
  以 $e$ 为底数的对数 $\log_e x$ 称为自然对数,通常写作 $\ln x$。只有正数才有自然对数,满足

  \tbox{$$\ln 1 = 0 $$}
  \tbox{$$\ln (xy) = \ln x + \ln y $$}
  \tbox{$$\ln (x/y) = \ln x - \ln y $$}
  \tbox{$$\ln (1/x) = -\ln x $$}    
\end{frame}

\begin{frame}

  \addfig{1}{mengli1.jpg}
  
  假设银行存款年利率为 $1\%$,你把一块钱存入银行,多少年后你的存款可以达到一万元?
\end{frame}



\begin{frame}

  \addfig{1}{mengli2.jpg}
  
  实际上,经济萧条,银行采用为 $-1\%$ 的负年利率,你把一万块钱存入银行,多少年后你的存款会小于一元?
\end{frame}



\begin{frame}
  \frametitle{知识点}
  \tbox{当 $|x|\ll 1$ 时,$$\ln(1+x) \approx x$$}
\end{frame}


\begin{frame}
 在函数曲线 $y=\ln x$ 上过点 $(a, \ln a)$ 作切线,切线的斜率是多少?
\end{frame}


\begin{frame}
  利用 $\ln\left(1+\frac{1}{n}\right) \approx \frac{1}{n} $,说明当 $n$ 很大时
  $$1+\frac{1}{2}+\frac{1}{3} + \ldots + \frac{1}{n} \approx \ln n$$
\end{frame}


\begin{frame}
  \frametitle{总结}
  \tbox{
  当 $|x|\ll 1$ 时,
  \bea
  (1+x)^\alpha & \approx & 1+\alpha x \newl
  \sin x & \approx & x \newl
  \cos x & \approx & 1-x^2/2 \newl
  e^x &\approx & 1+x \newl
  \ln (1+x) &\approx & x 
  \eea
  }
\end{frame}



\begin{frame}
  \frametitle{Homework}
  \bitem
\item{当 $x$ 很小时,我们希望作更精确的估算: $e^x \approx 1+x+cx^2$。利用 $e^x=e^{x/2}e^{x/2}$ 确定 $c$ 的值。}
\item{当 $x$ 很小时,我们希望作更精确的估算: $\ln(1+x) \approx x + cx^2 $。利用 $\ln \left[(1+x)^2\right] = 2\ln(1+x)$ 确定 $c$ 的值。}  
  \eitem
\end{frame}

\ech
\end{document}
