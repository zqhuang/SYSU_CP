\documentclass[CJK,13pt]{beamer}
\input{macros.tex}
\input{titlepage.tex}
  \date{}
  \begin{document}
  \bch
\tpage{1}{小量近似}



\begin{frame}
  \addfig{2}{xuebaliancheng.jpg}
  
  你大一因为没有学会《高x在xx中的应用》这门课,导致绩点不是很理想。大二,你奋发图强,把绩点提高了 2\%;大三继续努力,又把绩点提高了 3\%;大四还是忍住了没有浪,又把绩点提高了 4\%。问:你大四的绩点和大一相比提高了百分之多少?
\end{frame}


\begin{frame}
  
  $$  2\% + 3\% +4\% = 9\%$$

  \addfig{1}{shuxuetiyu.jpg}
  
\end{frame}


\begin{frame}

  $$  (1+2\%)(1+3\%)(1+4\%)-1 \approx 9\% $$

  \addfig{1}{lihai.jpg}
\end{frame}


\begin{frame}
  一个正方体的边长增大 $1\%$,则它的体积大约增加 ( )

  \skipline
  
  (A) 1\% (B) 2\% (C) 3\% (D) 4\%
\end{frame}



\begin{frame}
  \frametitle{知识点}
\tbox{当 $|x|\ll 1$ , 且 $|\alpha|$ 不太大时,
  $$(1+x)^\alpha \approx 1+\alpha x $$}

{\bf 注意:这里的 $\alpha$ “不能太大”,具体涵义我们之后会讨论。}
\end{frame}


\begin{frame}
  在函数 $y=x^3$ 对应的曲线上,过 $(a, a^3)$ 点作一条切线,切线的斜率等于多少?
\end{frame}


\begin{frame}
  一个保持直线运动的质点的位移 $s$ 随着时间 $t$ 变化的规律为
  $$s = \lambda t^3,$$
  这里的 $\lambda$ 为已知常量。计算在任意 $t$ 时刻质点的运动速率。
\end{frame}


\begin{frame}
  一个保持直线运动的质点的位移 $s$ 随着时间 $t$ 变化的规律为
  $$s = \lambda t^3,$$
  这里的 $\lambda$ 为已知常量。计算在任意 $t$ 时刻质点的加速度大小。
\end{frame}


\begin{frame}
  \frametitle{知识点}
\tbox{当 $|\theta| \ll 1$ 时, 
  $$ \sin \theta  \approx \theta \approx \tan \theta $$}

\addfig{2}{sintheta_approx.png}

$$OA=OB=1, AC = \sin\theta, BD = \tan\theta, \wideparen{AB} = \theta$$
\end{frame}


\begin{frame}
  
  三角函数常见公式:
  $$\sin(x+y) = \sin x \cos y + \cos x \sin y $$
  $$\cos(x+y) = \cos x \cos y - \sin x \sin y $$  
  $$\sin(2x) = 2\sin x \cos x$$
  $$\cos(2x) = 1-2\sin^2x = 2\cos^2x -1 $$

  \addfig{2}{kaowan.jpg}
\end{frame}


\begin{frame}
  描述中国古代的割圆术,并由此计算极限
  $$\lim_{n\rightarrow \infty} n \sin\frac{\pi}{n}\cos\frac{\pi}{n} $$
  的值。

  \addfig{3}{geyuanshu.jpg}

  \addfig{3}{geyuanshu.png}  
\end{frame}


\begin{frame}
  一维简谐振动的质点的位移 $s$ 随着时间 $t$ 变化的规律为
  $$s = A\sin{(\omega t)}$$
  这里的 $A,\omega$ 为已知常量。计算在任意 $t$ 时刻质点的速度和加速度的大小。
\end{frame}

\begin{frame}
  利用 $\cos\theta = 1-2\sin^2\frac{\theta}{2}$,说明如果 $|\theta|\ll 1$,则
  $$\cos\theta \approx 1- \frac{\theta^2}{2}$$
  (注意这是个二阶近似,比我们之前讨论的线性近似更精确)
\end{frame}  

\begin{frame}
  当 $\theta$ 很小时,我们希望作更精确的估算
  $$ \sin \theta \approx \theta + c\theta^3 $$
  (由于 $\sin \theta$ 是奇函数,不会有 $\theta^2$ 量级的修正)。请利用倍角公式证明 $c=-\frac{1}{6}$。
\end{frame}


\secpage{小量近似的静力学应用}{想知道TA在干什么,就轻轻地拉TA一下}

\begin{frame}
  \addfig{0.6}{Urope.png}

  如图,长度为 $\ell$的均匀柔软无弹性的细绳搭在光滑的半径为 $R$ 的圆柱面上。绳子上的最大张力是绳子重量的多少倍?
  
\end{frame}



\begin{frame}
  \frametitle{Homework}
  \bitem
\item[1]{把单摆的小角度摆动时的动能和势能写成二阶小量近似,并通过和一维简谐振动的对比写出单摆小摆幅摆动的周期公式。}
\item[2]{如图,质量为 $m$,长度为 $\ell$,均匀柔软无弹性的细绳悬挂在 $A$, $B$ 两点, $B$ 点比 $A$ 点高 $h$,已知 $A$ 点处绳子的张力为 $F_A$,计算 $B$ 点处绳子的张力。
  \lfig{1.5}{Vrope.png} }
  \eitem
\end{frame}

\ech
\end{document}
