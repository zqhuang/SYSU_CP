\documentclass[CJK,13pt]{beamer}
\input{macros.tex}
\input{titlepage.tex}
  \date{}
  \begin{document}
  \bch
  \tpage{11}{带参量的积分和多重积分}

  \secpage{带参量的积分}{$$\Gamma(x) := \int_0^\infty t^{x-1}e^{-t}dt$$}

  \begin{frame}
  \frametitle{含参量的积分:固定上下界}
  设$f$是已知的二元函数,$a,b$是常数,可以用含参量的积分来定义函数
  $$F(x) =\int_a^b f(x, t) dt.$$
  当{\bf 积分收敛性和光滑性很好时,可以交换积分和求导的次序}
  $$F'(x) =\int_a^b \frac{\partial f(x, t)}{\partial x} dt.$$  
\end{frame}


  \begin{frame}
    \frametitle{思考题}
    已知积分
    $$\int_{-\infty}^\infty e^{-x^2}dx = \sqrt{\pi}. $$
    计算
    $$\int_{-\infty}^\infty x^2 e^{-x^2}dx $$
    的值。
\end{frame}


  
\begin{frame}
  \frametitle{含参量的积分:变上下界}
  设$\alpha,\beta$是已知的一元函数,$f$是已知的二元函数,定义函数
  $$F(x) =\int_{\alpha(x)}^{\beta(x)} f(x, t) dt.$$
  当积分收敛性和光滑性很好时,
  \tbox{$$F'(x) =\int_{\alpha(x)}^{\beta(x)} \frac{\partial f(x, t)}{\partial x} dt + \beta'(x) f\left(x,\beta(x)\right) - \alpha'(x)f\left(x,\alpha(x)\right) .$$ }
  {\scriptsize (上面的结果有非常简单的几何解释,你能给出来吗?)}
  
\end{frame}

\thinka{在$x\in(0,\infty)$上定义$\Gamma$函数$$\Gamma(x) = \int_0^\infty t^{x-1}e^{-t}dt.$$
  试证明$-1<\Gamma'(1)<0.$}



  \secpage{多重积分}{$$\iint f(x,y) dxdy = \int \left[ \int f(x,y) dy\right] dx$$}
  
  \begin{frame}
    \frametitle{多重积分}
    多重积分:连续对多个变量计算积分的操作。

    例如二重积分:
    \bea
    \int_0^1dy \int_0^1dx\, (x+y)^2  &=& \int_0^1 \frac{1}{3}\left[(1+y)^3-y^3\right] \newl
    &=& \left.\frac{1}{12}\left[(1+y)^4-y^4\right]\right\vert^1_0 \newl
    &=& \frac{7}{6}
    \eea
  \end{frame}

  
  \thinka{请分别用一重积分和二重积分的方式计算半径为 $R$ 的圆的面积。}
  
  \begin{frame}
    \frametitle{曲面下的体积}
    \addfig{3.2}{volume_int2D.jpg}

    $$V = \iint f(x,y) dx dy$$
  \end{frame}


  \begin{frame}
    \frametitle{柱坐标系}
    \bmini{0.7}
    \addfig{2.5}{cylindricalcoor.png}
    \emini
    \bmini{0.26}
    和直角坐标的关系:
    
    \bea
    x &=& r\cos\theta \newl
    y&=& r\sin\theta \newl
    z &=& z
    \eea
    \emini
  \end{frame}


  \begin{frame}
  \frametitle{球坐标系}
  \bmini{0.7}  
  \addfig{2.5}{sphericalcoor.png}
  \emini
  \bmini{0.26}
  和直角坐标的关系:
    \bea
    x &=& r\sin\theta \cos\varphi\newl
    y&=& r\sin\theta\sin\varphi \newl
    z &=& r\cos\theta
    \eea
    \emini
  \end{frame}



  \thinkb{请分别用一重积分、二重积分和三重积分的方式计算半径为 $R$ 的球的体积。}  


  \begin{frame}
    \frametitle{Homework}
    \bitem
  \item[1]{你能否把 $\Gamma'(1)$ 的范围估算得更准确一些?}
  \item[2]{柱坐标系的 $r=$ 常数, $\theta=$ 常数,$z=$ 常数分别代表了什么曲面?}
  \item[3]{球坐标系的 $r=$ 常数, $\theta=$ 常数,$\varphi=$ 常数分别代表了什么曲面?}    
  \item[4]{计算质量为$m$,半径为 $R$ 的球绕自己的直径的转动惯量。}
    \eitem
  \end{frame}
  

\ech
\end{document}
