\documentclass[CJK,13pt]{beamer}
\input{macros.tex}
\input{titlepage.tex}
  \date{}
  \begin{document}
  \bch
  \tpage{12}{不等式}

%\left\vert\sum_i x_iy_i\right\vert\le \left(\sum_i |x_i|^p\right)^{1/p}\left(\sum_i |y_i|^q\right)^{1/q}
  
  \secpage{不等式入门知识}{上凸函数的平均值不大于平均值的上凸函数}

  \begin{frame}
    \frametitle{上凸函数和下凸函数}
    设函数 $f(x)$ 在 $[a,b]$ 内连续。如果对任意 $x,y\in[a,b]$ 有
    $$\frac{f(x)+f(y)}{2}\le f\left(\frac{x+y}{2}\right),$$
    则称 $f(x)$ 是 $[a,b]$ 内的上凸函数 (concave function)。

    如果上述定义中的不等号中的等号当且仅当 $x=y$ 时才能取到,则称 $f(x)$ 是严格上凸函数。
  \end{frame}


  \begin{frame}
    \frametitle{二阶可导的情况}
    如果函数 $f(x)$ 在 $[a,b]$ 内连续,在 $(a,b)$ 内二阶可导且二阶导数处处不大于零,则 $f(x)$ 是 $[a,b]$ 内的上凸函数;
    
    \addfig{2}{concave.png}
    
    (如果二阶导数处处小于零,则 $f(x)$ 是严格上凸函数。)
  \end{frame}
  


  \begin{frame}
    \frametitle{琴生不等式(Jensen's Inequality):二元情形}
    设$f(x)$ 是 $[a,b]$ 内的上凸函数,则对于任意 $\lambda\in(0,1)$ 以及 $x,y\in[a,b]$,有
    $$ \lambda f(x) + (1-\lambda)f(y) \le f\left(\lambda x + (1-\lambda) y\right).$$

    \addfig{3}{jensen.png}

    如果 $f$ 是严格上凸函数,则上述不等式中的等号当且仅当 $x=y$ 时才能取到。
    
  \end{frame}

  \begin{frame}
    \frametitle{琴生不等式}
    \tbox{
    设 $f(x)$ 是 $[a,b]$ 内的上凸函数,$n$ 为正整数。 $x_1,x_2,\ldots, x_n $ 是 $[a,b]$ 上的 $n$ 个实数; 正实数权重 $\lambda_1,\lambda_2,\ldots,\lambda_n$  满足总权重归一化($\sum_{i=1}^n\lambda_i=1$)。 那么一定有
    $$\sum_{i=1}^n \lambda_if(x_i) \le f\left(\sum_{i=1}^n\lambda_i x_i\right).$$
    如果$f(x)$是严格上凸函数,则上面不等式中的等号当且仅当所有的 $x_i$ 都相等时才能取到。
    }
  \end{frame}

  \thinka{用琴生不等式证明AM-GM不等式:
    \tbox{有限个非负实数 $a_1,a_2,\ldots, a_n$ 的几何平均不大于它们的算数平均
      $$\sqrt[n]{a_1a_2\ldots a_n} \le \frac{1}{n}\left(a_1+a_2+\ldots+a_n\right).$$
    等号当且仅当所有的 $a_i$ 相等时取到。}
  }

  \begin{frame}
    \frametitle{例题 (2019考研压轴题)}
      证明对任意正实数 $a,b,c$,不等式
      $$ ab^2c^3\le \frac{27}{16}\left(\frac{a+b+c}{3}\right)^6$$
      成立。
  \end{frame}

  
  \begin{frame}
    \frametitle{npy看得懂吗?}
    \addfig{2.5}{520.jpg}
  \end{frame}
    

  \begin{frame}
    \frametitle{$p$次均根的单调性}
    前面的AM-GM不等式其实只是诸多均值不等式的一种,一般地,有下列定理:
    \tbox{
    设 $a_1,a_2,\ldots, a_n$ 是正实数。对$p\ne 0$,定义它们的 $p$ 次均根为
    $$M(p) \equiv \left(\frac{1}{n}\sum_{i=1}^na_i^p\right)^{1/p};$$
    对 $p$ 等于零,则定义 0 次均根为几何平均
    $$M(0) \equiv \sqrt[n]{\prod_{i=1}^na_i}.$$
    那么$M(p)$ 是连续的单调不减函数;并且当 $a_1,a_2,\ldots, a_n$ 不全相同时,$M(p)$ 是连续的严格单调上升函数。
    }
  \end{frame}

  \begin{frame}
    \frametitle{例题 (2020考研压轴题)}
      已知正实数$x,y,z$ 满足 $x^2+y^2+z^2=a$。证明不等式
      $$ x^3+y^3+z^3 \ge \frac{a\sqrt{3a}}{3}$$
      成立。
  \end{frame}
  

  \begin{frame}
    \frametitle{柯西不等式(积的和的平方小于等于平方和的积)}
  \tbox{对任意 $2n$  个实数 $x_1,x_2,\ldots, x_n; y_1, y_2,\ldots, y_n$ 都有
    $$\left(\sum_{i=1}^n x_iy_i\right)^2 \le \left(\sum_{i=1}^nx_i^2\right) \left(\sum_{i=1}^ny_i^2\right).$$
   等号当且仅当 $x_1,x_2\ldots$ 和 $y_1, y_2,\ldots$ 成同样的比例关系时(即$x_iy_j = x_jy_i$ 对所有 $i,j$ 成立)取到。
  }
  \end{frame}
  
  
  \begin{frame}
    \frametitle{Holder不等式}
  \tbox{设正数 $p,q$ 满足 $\frac{1}{p}+\frac{1}{q}=1$,则对任意 $2n$  个非负实数 $x_1,x_2,\ldots, x_n; y_1, y_2,\ldots, y_n$ 都有
    $$\sum_{i=1}^n x_iy_i \le \left(\sum_{i=1}^nx_i^p\right)^{1/p} \left(\sum_{i=1}^ny_i^q\right)^{1/q}.$$
   等号当且仅当 $x_1^p,x_2^p\ldots$ 和 $y_1^q, y_2^q,\ldots$ 成同样的比例关系时(即$x_i^py_j^q = x_j^py_i^q$ 对所有 $i,j$ 成立)取到。
  }
  当 $p=q=2$ 时 Holder不等式退化为柯西不等式。
  \end{frame}


  \begin{frame}
    \frametitle{Minkowski不等式}
  \tbox{设 $p>1 $ ,则对任意 $2n$ 个非负实数 $x_1,x_2,\ldots, x_n; y_1, y_2,\ldots, y_n$ 都有
    $$\left(\sum_{i=1}^n (x_i+y_i)^p\right)^{1/p} \le \left(\sum_{i=1}^nx_i^p\right)^{1/p}+ \left(\sum_{i=1}^ny_i^p\right)^{1/p}.$$
    等号当且仅当$x_1,x_2\ldots$ 和 $y_1, y_2,\ldots$ 成同样的比例关系时(即$x_iy_j = x_jy_i$ 对所有 $i,j$ 成立)取到。
  }
  当 $p=2$ 时,Minkowski不等式退化为欧氏空间的“三角形两边之和不小于第三边”的结论。
  \end{frame}
  


  \secpage{积分不等式}{默认所有积分都存在}
  
  \begin{frame}
    \frametitle{绝对值的和不小于和的绝对值}
    \tbox{
    $$ \left\vert \int_a^b f(x)dx\right\vert \le \int_a^b |f(x)|dx $$
    }
  \end{frame}


  \begin{frame}
    \frametitle{柯西不等式}
    \tbox{$$\left[\int_a^b f(x)g(x)dx\right]^2 \le \left[\int_a^bf^2(x) dx \right]\left[\int_a^bg^2(x) dx\right]$$}    
  \end{frame}


  \begin{frame}
    \frametitle{Holder不等式}

    \tbox{设正数 $p, q$ 满足 $\frac{1}{p}+\frac{1}{q}=1$,则
      $$\left\vert\int_a^b f(x)g(x)dx\right\vert \le \left(\int_a^b|f(x)|^p dx\right)^{1/p}\left(\int_a^b|g(x)|^q dx\right)^{1/q}$$}    
  \end{frame}


  \begin{frame}
    \frametitle{Minkowski不等式}

    \tbox{设 $p>1$, 则{\small
      $$\left(\int_a^b |f(x)+g(x)|^pdx\right)^{1/p} \le \left(\int_a^b|f(x)|^p dx\right)^{1/p}+\left(\int_a^b|g(x)|^p dx\right)^{1/p}$$}}    
  \end{frame}
  


  \begin{frame}
    \frametitle{均值不等式}
    \tbox{设函数 $f(x) > 0$ 在 $[a,b]$ 内处处成立。对$p\ne 0$ 定义 $p$ 次绝对均根为
      $$M(p) \equiv \left(\frac{1}{b-a}\int_a^b\left(f(x)\right)^p dx\right)^{1/p}$$
    对 $p=0$ 定义
    $$M(0) = e^{\frac{1}{b-a}\int_a^b\ln f(x) dx}$$
    则 $M(p)$ 是连续单调不减函数。}
    
  \end{frame}
  

  \secpage{实用套路}{\addfig{1}{taolu.jpg}}
  
  \begin{frame}
    \frametitle{套路:用导函数积分表示的$p$次上界}
  \tbox{
  设$a<b$,$f(x)$ 在 $[a,b]$ 内连续,在$(a,b)$内可导,$f(a)=0$。则对任意 $x\in[a,b]$ 以及 $p>1$,有
  $$ |f(x)|^p \le (x-a)^{p-1}\int_a^x |f'(x)|^p dx $$
  }
  \end{frame}

  
  \begin{frame}
    \frametitle{例题 (2017考研压轴题)}
    设$a<b$,$f(x)$ 在 $[a,b]$ 内连续,在 $(a,b)$ 内可导,$f(a)=0$。证明:
    $$ \int_a^b |f(x)|^2 dx \le \frac{1}{2}(b-a)^2\int_a^b |f'(x)|^2 dx. $$   
  \end{frame}

  \begin{frame}
    \frametitle{例题 (2018考研压轴题)}
    设$a<b$,$f(x)$ 在 $[0,2]$ 内连续,在 $(0,2)$ 内可导,$f(0)=f(2)=0$。证明:
    $$ \int_0^2 |f(x)f'(x)| dx \le \frac{1}{2}\int_0^2 |f'(x)|^2 dx.$$   
  \end{frame}
  
  
  
  \begin{frame}
    \frametitle{Homework}
    \bitem
  \item{用直接配方的方法证明柯西不等式}
  \item{某种气体分子的质量为 $m$。一箱该气体的分子平均速率为$\bar{\upsilon}$。那么这些气体分子的平均动能和 $\frac{1}{2}m\bar{\upsilon}^2$ 哪个大?为什么?}
    \eitem
  \end{frame}
\ech
\end{document}
