\documentclass[CJK,13pt]{beamer}
\input{macros.tex}
\input{titlepage.tex}
  \date{}
  \begin{document}
  \bch
  \tpage{10}{隐函数的处理}

  \begin{frame}
    \frametitle{物理场景假设}
    在本讲中,我们仍将假设物理场景,多元函数足够光滑(任意阶偏导数存在且连续,这样偏导都可以交换次序)。如果多个变量 $x,y,\ldots$ 满足一个制约关系:
    $$f(x,y,\ldots) = 0$$
    那么就有
    $$ \hat{D} f = 0 $$
    这里的全微分算符  $\hat{D} = dx \frac{\partial}{\partial x} + dy \frac{\partial f}{\partial y} + \ldots $.
  \end{frame}

  \thinkb{
    设实变量 $x,y$ 满足
    $$ x^2y = e^x- e^y,$$
    当 $x$ 大约为多少时, $y$ 取到最小值?
  }


  \secpage{单自由度系统}{$$\frac{dx}{dy}\frac{dy}{dz}\frac{dz}{dx} = 1$$}
  
  \begin{frame}
    \frametitle{单自由度系统}
    假设有一堆变量 $X_1, X_2, X_3 \ldots$,当其中任意一个 $X_i$ 变动小量 $d X_i$ 时,其余所有变量的变化都被
    $$dX_j = c_{ij} dX_i $$
    确定 (其中 $c_{ij} = \frac{dX_j}{dX_i} $ 可以由 $X_1, X_2, \ldots $ 唯一确定)时,我们称该系统为{\blue 单自由度系统}。

    \skipline
    
    {\blue 对单自由度系统,可以把 $\frac{dX_j}{dX_i}$ 当成普通的商处理。}
  \end{frame}

  \begin{frame}
    \frametitle{例子}
    例如 $x$, $y=e^x$, $z=\arctan x$ 这三个量就是个单自由度系统。
    $$\frac{dy}{dx} = e^x = y$$
    $$\frac{dx}{dz} = 1+x^2 $$
    $$\frac{dz}{dy} = \frac{1}{(1+x^2)y} $$
  \end{frame}

\secpage{二自由度系统}{$$\left(\frac{\partial  X}{\partial Y}\right)_Z  \left(\frac{\partial  Y}{\partial Z}\right)_X \left(\frac{\partial  Z}{\partial X}\right)_Y = -1$$}

  \begin{frame}
    \frametitle{二自由度系统}
    假设有一堆变量 $X_1, X_2, X_3 \ldots$,当其中任意两个 $X_i$, $X_j$ 分别变动小量 $d X_i$, $dX_j$ 时,其余所有变量的变化都被
    $$dX_k = \alpha_{ki,j} dX_i + \alpha_{kj,i} dX_j$$
    确定 (其中每一个 $\alpha_{ki,j} = \left(\frac{\partial X_k}{\partial X_i}\right)_{X_j}$ 都可以由 $X_1, X_2, \ldots $ 唯一确定)时,我们称该系统为{\blue 二自由度系统}。

    \skipline

    我们的偏导符号发生了一点改变:{\blue $ \left(\frac{\partial X_k}{\partial X_i}\right)_{X_j}$ 表示当固定 $X_j$ 不变时, $X_k$ 和 $X_i$ 的微小改变量之比。} 之所以能这样说,是因为二自由度系统固定一个变量 $X_j$ 时,就成为单自由度系统,$X_k$ 和 $X_i$ 之比就可以唯一地由  $X_1, X_2, \ldots $ 确定。

    \skipline
    
    显然,根据定义即有{\blue  $\left(\frac{\partial X_k}{\partial X_i}\right)_{X_j} \left(\frac{\partial X_i}{\partial X_k}\right)_{X_j} = 1$}.
  \end{frame}


  \begin{frame}
    \frametitle{和以前的偏导符号的对比}
    在研究显式二元函数 $f(x,y)$ 时,其实 $f,x,y$ 这三个量也构成了二自由度系统。我们以前的符号 $\frac{\partial f}{\partial x}$,其实是默认自变量为 $x,y$,所以把 $\left(\frac{\partial f}{\partial x}\right)_y$的固定量下标 $y$ 给省略了。在讨论一般的二自由度系统时,并不清楚哪些量是自变量,所以这种固定量下标不能省略。

    \skiplines
    
    但在讨论二阶偏导时,写一堆固定量下标很麻烦(更主要是丑)。有时为了书写方便会事先约定哪两个量为自变量。这时就可以省略固定量下标。
  \end{frame}
  

\begin{frame}
  \frametitle{简单链式法则}
  \tbox{
    设三个变量$X$, $Y$, $Z$, $W$ 是二自由度系统的四个变量,则
    $$\pfrac XYW \pfrac YZW = \pfrac XZW $$
  }

  \skiplines

  证明:因为固定 $W$ 时,系统自由度降为1个,所以这简单就是一元函数微分的链式法则。
\end{frame}  


\begin{frame}
  \frametitle{循环偏微分乘积定理}
  \tbox{
    设三个变量$X$, $Y$, $Z$是二自由度系统的三个变量,则
    $$\left(\frac{\partial  X}{\partial Y}\right)_Z  \left(\frac{\partial  Y}{\partial Z}\right)_X \left(\frac{\partial  Z}{\partial X}\right)_Y = -1$$
  }

 {\small  证明:由二自由度系统的定义有 $dX = \pfrac XYZ dY +\pfrac XZY dZ$。然后考虑固定 $X$,也是 $dX=0$时, $dY$ 和 $dZ$ 就满足:
  $$ \pfrac XYZ dY +\pfrac XZY dZ = 0$$
  也就是说,固定 $X$ 时, $dY$ 和 $dZ$ 之比
  $$\pfrac YZX = - \frac{\pfrac XZY}{\pfrac XYZ}.$$
  再利用 $ \pfrac XZY \pfrac ZXY = 1$ 即得证。
  }
\end{frame}  

\begin{frame}
  \frametitle{切换固定量}
  \tbox{
    设$W, X, Y, Z$是二自由度系统的四个变量,则有
    $$ \pfrac WXY - \pfrac WXZ = \pfrac WZX \pfrac ZXY $$}

  {\small
    证明:根据二自由度系统的定义
    $$ dW = \pfrac WXZ dX + \pfrac WZX dZ $$
    在固定 $Y$ 时,利用 $ dZ = \pfrac ZXY dX$ 就可以得到固定 $Y$ 时 $dW$ 和 $dX$ 之比为
    $$\pfrac WXY = \pfrac WXZ + \pfrac WZX \pfrac ZXY $$
    
    }
  
\end{frame}


\secpage{单一成分平衡态系统}{$$dU = TdS-pdV $$
  $$dH = TdS+Vdp $$
  $$dF = -SdT-pdV $$
  $$dG = -SdT+Vdp $$  
}

\begin{frame}
  \frametitle{单一成分平衡态系统}
  在热学中,单一成分平衡态系统的压强 $p$、体积 $V$、热力学温度 $T$ 之间一般满足一个制约关系。这个制约关系可以用 $p$-$V$-$T$ 空间里的一个曲面表示,该曲面称为 $p$-$V$-$T$ 相图。

  \addfig{3}{PVTdiagram.png}
  
\end{frame}


\begin{frame}
  \frametitle{单一成分平衡态系统是二自由度系统}
  描述单一成分平衡态系统,不止可以用我们熟悉的 $p$, $V$, $T$ (相图给出它们之间的制约关系),还可以用另外一些:熵 $S$,内能 $U$ ,焓 $H\equiv U+pV$,自由能 $F\equiv U-TS$,自由焓 $G\equiv F+pV$。可以认为{\blue $p, V, T, S, U, H, F, G$ 这些“态函数”构成了二自由度系统}。
\end{frame}

\begin{frame}
  \frametitle{思考题}
单一成分平衡态物质的膨胀系数定义为
  $$\alpha \equiv \frac{1}{V}\pfrac VTp$$
  压强系数定义为
  $$\beta \equiv \frac{1}{p}\pfrac pTV$$
  压缩系数定义为
  $$\kappa \equiv -\frac{1}{V}\pfrac VpT$$
  试证明:
  $$\alpha = \kappa \beta p$$
\end{frame}




\begin{frame}
  \frametitle{热力学第一定律}
  由于准静态过程中吸热量$\dbar Q$ 和熵变 $dS$ 存在 {\blue $\dbar Q=TdS$ }的关系,热力学第一定律(能量守恒)给出
  {\blue $$dU = TdS - p dV.$$}
  然后根据 $H, F, G$ 的定义,有
  {\blue
    \bea
    dH &=& TdS + Vdp \newl
    dF &=& -SdT - pdV \newl
    dG &=& -SdT + Vdp
    \eea
  }
  这些方程是描述单一成分平衡态系统的基本方程,在学习热力学时将反复强调。
  
\end{frame}

\begin{frame}
  \frametitle{麦克斯韦关系}
  对单一成分平衡态系统证明下述Maxwell关系:
\bitem
\item{$\pfrac TVS = -\pfrac pSV$}
\item{$\pfrac TpS = \pfrac VSp$}
\item{$\pfrac SVT = \pfrac pTV$}
\item{$\pfrac SpT = -\pfrac VTp$}
\eitem
\end{frame}  


\begin{frame}
  \frametitle{定体热容和定压热容}
  固定体积时,单位温度变化需要吸收的热量称为定体热容,用 $C_V$ 表示。也就是说,
{\blue  $$C_V := T\pfrac STV. $$}

  \skipline
  
  固定压强时,单位温度变化需要吸收的热量称为定压热容,用 $C_p$ 表示。也就是说,
  {\blue $$ C_p := T\pfrac STp.$$}
  请证明:
  $$ C_p - C_V = T\pfrac pTV \pfrac VTp $$
\end{frame}


\begin{frame}
  \frametitle{Homework}
  \bitem
\item[1]{$y\in[-\frac{\pi}{2},\frac{\pi}{2}]$ 以及 $x=\sin y$ 构成了单自由度系统。由此计算反正弦函数 $y=\arcsin x$ 的导函数。}
\item[2]{对单一成分平衡态系统证明 $$\pfrac HVp = T\pfrac pTS $$}
\item[3]{对单一成分平衡态系统证明 $$\pfrac UpV = -T\pfrac VTS$$}
\item[4]{对单一成分平衡态系统证明 $$\pfrac TSH +\frac{T^2}{V} \pfrac VHp = \frac{T}{C_p} $$
  }  
  \eitem
\end{frame}

\ech
\end{document}
