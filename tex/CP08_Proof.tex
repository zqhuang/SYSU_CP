\documentclass[CJK,13pt]{beamer}
\input{macros.tex}
\input{titlepage.tex}
  \date{}
  \begin{document}
  \bch
  \tpage{8}{泰勒公式和其他类似定理的证明}

  \begin{frame}
    \frametitle{证明题大概就是……}
    \addfig{3}{fuxiguake.jpg}
  \end{frame}

\secpage{“入门级”问题}{\addfig{2}{fangqi.jpg}}
  
\begin{frame}
  \frametitle{介值定理}
  
  设函数 $f(x)$ 在 $[a,b]$ 上连续,$f(a) < f(b)$。证明:对任意 $c\in \left(f(a), f(b)\right)$,存在 $\zeta\in(a,b)$ 使得 $f(\zeta)=c$。
\end{frame}

  
  \begin{frame}
    \frametitle{罗尔定理}
  设函数 $f(x)$ 在 $[a,b]$ 上连续,在 $(a,b)$ 内可导,且 $f(a)=f(b)$,证明:存在 $\zeta\in(a,b)$ 使得 $f'(\zeta)=0$。
\end{frame}

\secpage{套路1}{三人行,必有两个空档}

\begin{frame}
  设$a<b<c$,函数 $f(x)$ 在 $[a,c]$ 上连续,在 $(a,c)$ 内二阶可导,且 $f(a)=f(b)=f(c)$,证明:存在 $\zeta\in(a,c)$ 使得 $f''(\zeta)=0$。
\end{frame}


\begin{frame}
  设$a<b<c<d$,函数 $f(x)$ 在 $[a,d]$ 上连续,在 $(a,d)$ 内三阶可导,且 $f(a)=f(b)=f(c)=f(d)$,证明:存在 $\zeta\in(a,d)$ 使得 $f'''(\zeta)=0$。
\end{frame}

\begin{frame}
  \addfig{2}{peibushang.jpg}
\end{frame}


\secpage{套路2}{百搭款端点}

\begin{frame}
  设函数 $f(x)$ 在 $[a,b]$ 上二阶可导,且 $f(a)=f(b)$, $f'(a) = 0$,证明:存在 $\zeta\in(a,b)$ 使得 $f''(\zeta)=0$。
\end{frame}


\begin{frame}
  设函数 $f(x)$ 在 $[a,b]$ 上三阶可导,且 $f(a)=f(b)$, $f'(a) = f''(a) = 0$,证明:存在 $\zeta\in(a,b)$ 使得 $f'''(\zeta)=0$。
\end{frame}


\secpage{套路3}{动静结合}

\begin{frame}
  设函数 $f(x)$ 在 $[a,b]$ 上连续,在 $(a,b)$ 内可导。证明: 存在 $\zeta\in(a,b)$,使得 $f'(\zeta) = \frac{f(b)-f(a)}{b-a}$。
\end{frame}


\begin{frame}
  设函数 $f(x)$和 $g(x)$ 均在 $[a,b]$ 上连续,在 $(a,b)$ 内可导,$g(a)\ne g(b)$。证明: 存在 $\zeta\in(a,b)$,使得 $\frac{f(b)-f(a)}{g(b)-g(a)} = \frac{f'(\zeta)}{g'(\zeta)}$。
\end{frame}


\begin{frame}
  设函数 $f(x)$ 在 $[a,b]$ 上 $n+1$ 次可导。证明: 存在 $\zeta\in(a, b)$,使得
  $$ f(b) = \sum_{k=0}^n \frac{f^{(k)}(a)}{k!}(b-a)^k + \frac{f^{(n+1)}(\zeta)}{(n+1)!}(b-a)^{n+1} .$$
\end{frame}


\begin{frame}
  设函数 $f(x)$ 在 $[a,b]$ 上连续,在 $(a,b)$ 内二阶可导,且 $f(a)=f(b)=0$,证明:对每个 $x\in (a,b)$,都存在 $\zeta\in(a,b)$,使得 $f(x)=\frac{f''(\zeta)}{2}(x-a)(x-b)$.
\end{frame}


\begin{frame}
  \frametitle{Homework}
  \bitem
\item[1]{设$a<b<c$,函数 $f(x)$ 在 $[a,c]$ 上三阶可导,且 $f(a)=f(b)=f(c)$, $f'(a) = 0$,证明:存在 $\zeta\in(a,c)$ 使得 $f'''(\zeta)=0$。}
\item[2]{设函数 $f(x)$ 在 $[0,\pi]$ 上连续,在 $(0,\pi)$ 内二阶可导,且 $f(0)=f(\pi)=0$,证明:对每个 $x\in (0,\pi)$,都存在 $\zeta\in(0,\pi)$,使得 $f(x)= - \frac{\sin x}{\sin \zeta}f''(\zeta)$.
}
\item[3]{设 $f$ 在 $[a,b]$ 上处处可导,证明导函数$f'(x)$ 满足介值定理: 对介于 $f'(a)$ 和 $f'(b)$ 之间的 $c$,一定存在 $\zeta\in [a,b]$ 使得 $f'(\zeta)=c$.}  
  \eitem
\end{frame}

\ech
\end{document}
